\documentclass[a4paper]{article}
\usepackage[utf8]{inputenc}
\usepackage[english]{babel}
\usepackage{moreverb}
\usepackage{graphicx}
\title{Project Report \\ EDA095 Network Programming}
\date{\today}
\author{Fredrik Paulsson \\ dat11fp1@student.lu.se \and Jonas Jacobsson \\ dat11jja@student.lu.se \and Artur Matulaniec \\ dat11ama@student.lu.se \and Magnus Törnquist \\ ada08mto@student.lu.se}
%\setcounter{secnumdepth}{5}
%\setcounter{tocdepth}{5}
\begin{document}
\maketitle
%\tableofcontents

\section{Introduction}
The project in this course did not contain a specific task that was to be done. Rather the only demand was that the content of the course was to be of central focus in the project.

Since the course was about network programming we decided that our project were to be based on video streaming. Basically we wanted to stream video from a server to a client. At first we wanted to implement the server part using well-known protocols such as RTSP/RTCP/RTP and then connect our server to an already existing client such as VLC.

However, later on we decided to skip these existing protocols and implement our own client/server solution using a simple protocol that we ourselves defined.

This report will serve as documentation of our project work.

\section{Implementing a server using RTSP}
Här förklarar vi varför vi bytte fokus till att göra vår egen lösning...

\section{Features/Requirements}
Vilka krav vi har och/eller features

\section{Structure/Model}
Our solution consists of two parts, a server and a client. These are two independent programs that communicates using a protocol that we have oursleves defined. The communication uses a TCP connection.

\subsection{Library}
Här beskriver vi xuggle lite kortfattat, lägg gärna till referenser. Problemen med xuggle tykcer jag vi tar i diskussionen

\subsection{Client}
Viktigaste klasser och annat tekniskt hur klienten fungerar

\subsection{Server}
samma här fast för servern

\subsection{Protocol}
beskrivning av protokollet

\section{User Guide}
manual

\section{Evaluation/Discussion}
kolla rapportgrejsen som jag länkat till: http://fileadmin.cs.lth.se/cs/Education/EDA095/2010/projektrapport.pdf
\section{Source Code}
Vi beskriver bara hur källkoden är organiserad, för mycket att inkludera allt




\begin{thebibliography}{1}
\bibitem{wikipedia}
http://en.wikipedia.org
\end{thebibliography}
\end{document}
